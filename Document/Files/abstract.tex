\begin{center}
	\textbf{\centering \large Resumen} \vspace{0.5cm}\\
	\begin{minipage}{0.85\linewidth}
		La radiación solar participa en múltiples procesos biológicos y atmosféricos indispensables para la vida en la Tierra. La energía solar desencadena un gran número de reacciones y en las grandes ciudades interviene en la fotoquímica del smog. La variación de la intensidad solar depende en general de la composición atmosférica, ubicación geográfica, época del año y de la hora del día. Determinar la irradiancia solar global nos permite conocer su disponibilidad para beneficio humano, extendiendo su uso a través de la conversión y almacenamiento e incluso se puede utilizar en tratamientos médicos. En este contexto, identificar y caracterizar los elementos que la atenúan es esencial. Las nubes juegan un rol fundamental en el balance radiativo y dependiendo de su altura y estructura pueden dispersar, absorber o reflejar los fotones del sol. La cobertura de nubes es clave para cualquier tipo de pronóstico meteorológico, siendo la radiación solar un detector natural de las mismas. En las últimas décadas se han desarrollado una variedad de modelos para clasificar los diferentes tipos de cielo según las condiciones de nubes o porcentaje de nubosidad. Estos modelos son comparados con mediciones in situ o datos satelitales. Los niveles de complejidad y parámetros de entrada para estimar la irradiancia solar global en cielo despejado, con frecuencia se basan en expresiones empíricas. En otros casos, se requieren parámetros geométricos como el ángulo cenital o parámetros meteorológicos básicos como horas de sol, humedad relativa, presión, índice de claridad y temperatura. Algunos modelos incluyen parámetros de las condiciones atmosféricas como la profundidad óptica del aerosol, cantidad de agua precipitable y columna de ozono. Los datos de cielo despejado suelen extraerse del conjunto de datos de radiación solar medidos en todos los tipos de cielo, a menudo mediante el uso de algoritmos que se basan en otros parámetros meteorológicos medidos. Los procedimientos actuales para la extracción de datos de cielo despejado se han examinado y comparado entre sí para determinar su confiabilidad y dependencia de la ubicación. Se implementaron modelos basados en geometría, topología de datos y redes neuronales para determinar si un día tiene la condición de cielo despejado, parcialmente nublado ó nublado basado en las diferencias o proporciones entre las mediciones diarias in situ y los modelos de irradiancia solar extraterrestre y Robledo-Soler. Obteniendo que los modelos de redes convolucionales y bosques aleatorios entranados con las proporciones de las mediciones diarias y el modelo Robledo-Soler muestran mejor rendimiento.
	\end{minipage}
\end{center}
