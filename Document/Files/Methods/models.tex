\section{Modelos de irradiancia solar}
Los modelos de irradiancia solar pueden estimar el valor de irradiancia solar bajo condiciones de cielo despejado\cite{Gueymard_2012,Perez_Burgos_2017,Ineichen_2016}. Las estimaciones obtenidas por los modelos son usados para realizar comparaciones sobre mediciones de una locación, las cuales pueden contener datos inexistences o ruido. Existen diversos modelos basados en redes neuronales donde a partir de parámetros geoespaciales ó datos meteorológicos estiman el promedio por hora, dia, mes o año\cite{Hasni_2012,Kumar_2019,Ozgoren_2012,Sahan_2016}. Estos modelos requieren que la información de entrada sea detallada, por lo que representa un problema, aunado a esto, los modelos estan delimitados por la precisión que tienen sus estimaciones\cite{Ruiz_Arias_2018}. En este trabajo se propone el uso de modelos simples que pueden adaptarse a la localidad pero con la suficiente precisión para estimar la irradiancia globar bajo condiciones de cielo despejado.

\subsection{Irradiancia solar extraterrestre}
El modelo de irradiancia solar extraterrestre (GHI$_0$) esta definido de la siguiente manera\cite{Muhammad_1983}.

\begin{equation}
	\text{GHI}_0 = I_{SC}\left[ 1-0.033 cos\left( \frac{360n}{365}  \right)\right] cos(z)  \label{eq:GHI0}
\end{equation}

Donde I\textsubscript{SC} la constante solar con valor de 1367 W/m\textsuperscript{2}, n es el día consecutivo del año (n=1 es el primer día año y 365 es el último día del año, para años bisiestos el denominador cambia a 366 y el último día se toma como 366) y $z$ es el ángulo zenital definido de la  siguiente manera:

\begin{equation}
	cos(z) = cos(\phi)cos(\delta)cos(\omega)+sin(\phi)sin(\delta)
\end{equation}

Donde $\phi,\delta,\omega$ son la latitud, declinación solar y el ángulo solar de la locación y la hora local.

\subsection{Declinación y ángulo solar}
La dependencia en el tiempo en la ecuación \ref{eq:GHI0} se introduce por medio de la declinación solar (ecuación \ref{eq:declination}) y el ángulo solar (ecuación \ref{eq:angle_solar}), donde h\textsubscript{LTC} es la hora local y $\gamma$ es la fracción de rotación de la tierra con respecto al sol.

\begin{equation}
	\delta = 24.45 sin(\gamma)
	\label{eq:declination}
\end{equation}
\begin{equation}
	\omega = 15(h_{LTC}-12)
	\label{eq:angle_solar}
\end{equation}


\subsection{Irradiancia solar global horizontal}
Kwarikunda\cite{Kwarikunda_2021} menciona que realizaron comparaciones entre los modelos Berger-Duffle (BD), Adnot-Bouges-Campana-Gicquel (ABCG) y Robledo-Soler (RS) para obtener el modelo que realiza una estimación más cercana a las mediciones realizadas en diferentes locaciones con el piranómetro CMP10, el cual realiza mediciones en el rango 285-2800 nm. El modelo RS es el que obtiene una mejor estimación de la irradiancia solar a nivel del suelo. El modelo RS se encuentra definido en la ecuación \ref{eq:rs_model}.

\begin{equation}
	GHI_{RS} = a(cos z)^b exp(-c(90-z))
	\label{eq:rs_model}
\end{equation}

donde $cos z$ es el angulo zenital y $a,b,c$ son constantes a determinar. En la tabla \ref{table:rs_parameters} se encuentran los parámetros usados para el modelo.

\begin{table}[H]
	\centering
	\begin{tabular}{llll} \hline
		\textbf{Parametro } & \textbf{a} & \textbf{b} & \textbf{c}    \\ \hline
		Valor               & 1119       & 1.19       & 1x10$^{-6  }$ \\ \hline
	\end{tabular}
	\caption{Parámetros del modelo RS.}
	\label{table:rs_parameters}
\end{table}
