\section{Sistema de Monitoreo Ambiental}
El Área Metropolitana de Monterrey (AMM) se ubica en una región montañosa donde se realizan extracciones de material para la construcción (pedreras) a la par de actividades industriales y alto flujo vehicular. El Sistema de Monitoreo Ambiental (SIMA) tiene como objetivo evaluar la calidad del aire, monitoreando las concentraciones de contaminantes atmosféricos a las que se encuentra expuesta la población del AMM. Cuando el AMM se encuentra con altos índices de contaminación atmosférica que representa un peligro para el ser humano, el SIMA es el responsable de reportar estos periodos. El SIMA se compone de 13 estaciones de monitoreo repartidas a lo largo del AMM (figura \ref{fig:stations}).
\begin{figure}[H]
	\centering
	\includegraphics[width=14cm]{Graphics/map_nolabels.eps}
	\caption{Ubicación geográfica de las estaciones meteorológicas del SIMA en el AMM.}
	\label{fig:stations}
\end{figure}
En la tabla \ref{table:stations_information} se muestra la información geográfica de las estaciones del SIMA en el AMM.
\begin{table}[H]
	\footnotesize
	\centering
	\begin{tabular}{llcrrr} \hline
		\textbf{Ciudad}          & \textbf{Nombre} & \begin{tabular}{cc}														 \textbf{Elevación} \\\textbf{(m s. n. m.)}\end{tabular} & \textbf{Latitud (°N)} & \textbf{Longitud (°O)} \\ \hline
		Guadalupe                & Sureste         & 492                                                                                       & 25.67                 & -100.25                \\
		Monterrey                & Centro          & 560                                                                                       & 25.67                 & -100.34                \\
		Monterrey                & Noroeste        & 571                                                                                       & 25.76                 & -100.37                \\
		San Nicolas de los Garza & Noreste         & 476                                                                                       & 25.75                 & -100.26                \\
		Santa Catarina           & Suroeste        & 694                                                                                       & 25.68                 & -100.46                \\
		Garcia                   & Noroeste2       & 716                                                                                       & 25.78                 & -100.59                \\
		Escobedo                 & Norte           & 528                                                                                       & 25.80                 & -100.34                \\
		Apodaca                  & Noreste2        & 432                                                                                       & 25.78                 & -100.19                \\
		Juarez                   & Sureste2        & 387                                                                                       & 25.65                 & -100.10                \\
		San Pedro Garza Garcia   & Suroeste2       & 636                                                                                       & 25.66                 & -100.41                \\
		Cadereyta de Jimenez     & Sureste3        & 340                                                                                       & 25.58                 & -99.99                 \\
		Monterrey                & Sur             & 630                                                                                       & 25.57                 & -100.25                \\
		San Nicolas de los Garza & Norte2          & 520                                                                                       & 25.73                 & -100.31                \\ \hline
	\end{tabular}
	\caption{Información de la localización geográfica de las estaciones del SIMA en el AMM.}
	\label{table:stations_information}
\end{table}
La base de datos que contiene mediciones de irradiancia solar por hora en las estaciones del SIMA en el periodo 1993-2021. Se realizó un filtro de los datos que cumplen las siguientes condiciones:
\begin{itemize}
	\item Un dato diario es válido cuando contiene al menos 10 mediciones entre las 8 a las 19 horas.
	\item Un mes es considerado cuando contiene al 21 datos diarios validados.
\end{itemize}
En la figura \ref{fig:distribution_data} se muestra la distribución de los meses en las estaciones del SIMA que cumplen las condiciones de selección.
\begin{figure}[H]
	\centering
	\includegraphics[width=13cm]{Graphics/Distribution_stations.png}
	\caption{Distribución de los meses validos para las estaciones meteorológicas del SIMA en el periodo 1993-2021.}
	\label{fig:distribution_data}
\end{figure}
\section{Creación de la base de datos}
Se seleccionaron las estaciones noroeste, noreste, sureste2 y suroeste en el periodo 2019-2021. Estas estaciones fueron elegidas debido a que la topografía es representativa y se encuentra en zonas estratégicas de diferentes fuentes de emisión. Adicionalmente cuentan con un gran número de mediciones dentro del periodo seleccionado. Con base a las mediciones diarias de cada estación se clasificaron de acuerdo al comportamiento de su intensidad solar en función de las horas del día. Las condiciones de cielo contempladas son: despejado, parcialmente nublado y nublado.
\subsection{Criterios para las condiciones de cielo despejado}
Los criterios para la clasificación de cielo despejado de manera predeterminada de las mediciones diarias son las siguientes:
\begin{itemize}
	\item Cielo despejado
	      \begin{itemize}
		      \item Un día de cielo despejado se caracteriza por tener un comportamiento gaussiano a lo largo del día, teniendo como máximo el mediodía solar. Para el AMM, el mediodía solar debe encontrarse alrededor de las 12:30-14:30 horas. Las mediciones deben registrar un valor diferente a cero entre las 6 a las 20 horas.
	      \end{itemize}
	\item Cielo parcialmente nublado
	      \begin{itemize}
		      \item Un día de cielo parcialmente nublado se caracteriza por presentar el comportamiento de un día con cielo despejado pero en ciertos intervalos de tiempo. Esto puede ocurrir en solo una hora, o en varias. Si el día contiene entre uno y cinco mediciones que caracterizan a un día despejado, entonces el día será clasificado como parcialmente nublado.
	      \end{itemize}
	\item Cielo nublado
	      \begin{itemize}
		      \item Un día nublado se caracteriza por presentar un comportamiento caótico o un comportamiento gaussiano con variaciones abruptas en intervalos de tiempo cortos.
	      \end{itemize}
\end{itemize}
En la figura \ref{fig:example_sky_conditions} se presentan diferentes mediciones donde se presentan las diferentes condiciones de cielo.
\begin{figure}[H]
	\centering
	\includegraphics[width=12cm]{Graphics/example_sky_conditions.png}
	\caption{Ejemplos de las clasificaciones de las condiciones de cielo a partir de mediciones diarias de cada una de las estaciones del SIMA.}
	\label{fig:example_sky_conditions}
\end{figure}
\subsection{Distribución de los datos}
En la figura \ref{fig:distribution} se muestra la distribución de las condiciones de cielo clasificadas de manera predeterminada, en esta se observa que existe una mayor cantidad de dias nublados en comparación de los dias categorizados como despejado o parcialmente nublado.
\begin{figure}[H]
	\centering
	\includegraphics[width=10cm]{Graphics/distribution_classification.png}
	\caption{Distrubución de las clasificaciones de las condiciones de cielo en la base de datos.}
	\label{fig:distribution}
\end{figure}
\subsection{Operaciones de comparación}
Se implementaron dos operaciones para comparar las diferencias y proporciones (ecuación \ref{eq:diff} y \ref{eq:ratio} respectivamente) entre las mediciones y los modelos GHI$_0$ y RS.
\begin{equation}
	d_t = \begin{cases}
		\text{Modelo} - \text{Medición} & \text{si Modelo}\neq 0 \\
		0                               & \text{si Modelo} = 0
	\end{cases}
	\label{eq:diff}
\end{equation}
\begin{equation}
	k_t = \begin{cases}
		\frac{\text{Medición}}{\text{Modelo}} & \text{si Modelo}\neq 0 \\
		0                                     & \text{si Modelo} = 0
	\end{cases}
	\label{eq:ratio}
\end{equation}
\subsection{Datos atípicos}
Los datos de las estaciones del SIMA pueden contener ruido o mediciones que físicamente no son posibles, a estos datos los denominamos como atípicos. Se implemento un filtrado de datos automática. El filtro de datos consiste en realizar una operación de comparación para cada medición (ecuación \ref{eq:ratio}) con respecto al modelo GHI, si para alguna hora se obtiene un valor mayor a 0.9, entonces la medición corresponde a un dato atípico y se eliminará de la base de datos. Si el modelo GHI$_0$ es igual a 0, entonces se sobreescribe la medición con el valor 0, esto con el propósito de eliminar el ruido que puede tener el radiometro de la estación analizada. En la figura \ref{fig:example_clean_data} se visualiza el proceso de filtrado de datos atípicos en dos fechas diferentes. Los valores atípicos ocurren a menudo al inicio o al final del día solar.
\begin{figure}[H]
	\centering
	\includegraphics[width=12cm]{Graphics/example_clean_data.png}
	\caption{Mediciones de irradiancia solar de la estación noroeste originales (izquierda) y sin valores atípicos (derecha).}
	\label{fig:example_clean_data}
\end{figure}
\subsection{Reconstrucción}
A partir de los datos filtrados, se aplicó un proceso de reconstrucción. El proceso de reconstrucción consiste en asignar el valor del promedio horario de las 10 primeras mediciones que tengan mayor semejanza a la medición a reconstruir. Se toman unicamente las mediciones de la misma estación en una ventana de tres meses (un mes anterior, el mes actual y el siguiente). La semejanza se calcula como:
\begin{equation}
	sim(m_i , m_j ) = \frac{m_i \cdot m_j}{||m_i|| ||m_j||}
	\label{eq:cosine}
\end{equation}
En la figura \ref{fig:restoration} se muestran los datos restaurados para los casos presentados en la figura \ref{fig:example_clean_data}.
\begin{figure}[H]
	\centering
	\includegraphics[width=12cm]{Graphics/example_restoration.png}
	\caption{Restauracion de mediciones por medio de promedios horarios de las 30 mediciones más semejantes al día seleccionado}
	\label{fig:restoration}
\end{figure}
Con las mediciones restauradas se realizaron las comparaciones (ecuación \ref{eq:diff} y ecuación \ref{eq:ratio}) con respecto a los modelos GHI$_0$ y RS.
