\section{Introducción}

La irradiancia solar a nivel de suelo esta influenciada por las condiciones  de cielo.  El índice de claridad de cielo (sky clearness index) es la proporción entre la radiación global en la superficie y la radiación solar extraterrestre\cite{Muhammad_1983}. Este índice puede ser un gran estimador para cuantificar una observación a lo largo del tiempo. Se ha exhibido que existe una caracterización basada en el índice de claridad cuando se tiene una resolución de los datos menor a 5 minutos\cite{Suehrcke_1988,Skartveit_1992,Jurado_1995}, la cual es capaz de distinguir de manera precisa la condición de cielo binaria (despejado o nublado). Para la clasificación diaria basada en radiación solar a nivel de suelo existen tres clases: despejado, parcialmente nublado y nublado. Maafi\cite{Maafi_2003} empleó la dimensión fractal aplicada a señales y Harrouni\cite{Harrouni_2005} implementó un análisis fracccionario con una resolución de 10 minutos para realizar la clasificación diaria de las condiciones de cielo. En este trabajo se empleará una resolución de 1 hora para mediciones de irradiancia global a nivel de suelo implementando modelos basados en geometría, topología de los datos y redes neuronales para realizar la clasificación de las condiciones de cielo diarias.
